\documentclass[a4paper,12pt]{article}
\usepackage[brazil]{babel}
\usepackage[utf8]{inputenc}
\usepackage[T1]{fontenc}
\usepackage{amsmath}
\pagestyle{headings}
\author{Julio Cesar Eiras Melanda}
\title{Computação Bio Inspirada \\
      Relatório da Resolução do Problema do Caixeiro
      Viajante com Algoritmos Genéticos}
\begin{document}
  \maketitle
  \tableofcontents
  \section{Introdução}
  O problema do \emph{Caixeiro Viajante} é um dos mais conhecidos problemas
  NP-completos da computação, e consiste em descobrir o menor caminho para
  percorrer um conjunto de cidades completamente conectadas.

  Um problema que pode ser resolvido facilmente na força bruta quando se tem
  poucas cidades, mas que fica facilmente inpossível de tratar quando o número
  de cidades cresce para algumas dezenas.

  Vamos exemplificar a complexidade do problema com números.

  Quando temos um conjunto de 5 cidades, temos $5!$ opções, ou seja, 120
  caminhos que podem ser a resposta para o problema.

  Já com 10 cidades, passamos a $10!$ opções, o que significa 3682800 opções, o
  que ainda é aceitável computacionalmente. Mas aumentemos este número para 15
  cidades e teremos $15!$ cidades o que nos dá o incrível número 1307674368000.

  Se explodirmos este cálculo para 20 cidades ou mais, o número de opções é tão
  grande que o tempo necessário para calcular o caminho ótimo ultrapassa a
  casa dos milhares de anos.
  \footnote{Considerando uma máquina de 1GHz, e imaginando que
  a distância de uma solução possa ser calculada em 100 ciclos, temos:
  2432902008176640000 soluções, que podem ser calculadas em
  aproximadamente 24329020081766 segundos, cerca de 67580611
  horas, que aproximamos em 2815858 dias, o que nos dá 7714 anos}

  Assim, surge a necessidade de encontrarmos algoritmos que consigam resolver
  este problema sem percorrer todas as soluções possíveis para encontrar a
  solução ótima. Uma das soluções mais usadas, consiste na utilização de
  algoritmos evolucionários, especialmente os algoritmos genéticos para a
  encontrar a solução ótima, ou alguma solução bastante próxima da ótima.

  \subsection{ALgoritmos Genéticos}
  Algoritmos Genéticos são um conjunto de Algoritmos Evolucionários que se
  inspiram na forma como a seleção natural escolhe os indivídos cujo código
  genético seja mais favorável à adaptação ao ambiente onde se encontra para
  encontrar soluções para problemas computacionalmente complexos.

  Para isto, as possíveis soluções ao problema devem ser modeladas de modo a
  serem representadas como um conjunto de cromossomos. Através de operadores
  genéticos, as soluções são testadas segundo uma função de aptidão, e
  das soluções mais bem adaptadas algumas são submetidas a cruzamentos para
  gerar novas populações, assim, a população de soluções tende a convergir para
  os valores ótimos.

  \section{Modelagem do Problema}
  \subsection{A distância entre as cidades}
  Para modelarmos nosso problema vamos representar as distâncias entre as
  cidades como uma matriz, onde cada linha e coluna representa uma cidade e as
  células representam a distância entre estas cidades. Assim, podemos considerar
  que num grupo de 5 cidades tenhamos as seguintes disâncias:
  \begin{displaymath}
    \begin{bmatrix}
      0 & 1 & 10 & 5 & 10\\
      1 & 0 & 2 & 10 & 6\\
      10 & 2 & 0 & 1 & 11\\
      5 & 10 & 1 & 0 & 3\\
      10 & 6 & 11 & 3 & 0
    \end{bmatrix}
  \end{displaymath}
  Onde o elemento $a_{ij}$ representa a distância da cidade $i$ à cidade $j$.

  \subsection{As possíveis soluções}
  Com as cidades e suas distâncias representadas como uma matriz, temos então as
  cidades representadas pelos números de 0 a 4, que são os índices das
  linhas/colunas da matriz.
  Logo, podemos representar o caminho percorrido como sendo uma sequência de
  inteiros, que podemos armazenar numa lista.

  Assim, as possíveis soluções para o problema serão listas de inteiros, fazendo
  desta uma boa representação para os cormossomos em nosso algoritmo genético,
  assim como na lista a seguir: \verb+[3, 0, 4, 2, 1]+

  \subsection{A função de aptidão}
  No problema do caixeiro viajante, os menores caminhos são melhores que os
  maiores caminhos, assim, podemos considerar nossa função de aptidão como sendo
  simplesmente o somatório das distâncias entre as cidades na ordem especificada
  em nossas soluções.

  Assim, considerando a matriz e a lista mostrados acima, teríamos como valor de
  aptidão o seguinte:
  \begin{align*}
    a_{30}=5 \\
    a_{04}=5 \\
    a_{42}=1 \\
    a_{21}=2 \\
    a_{10}=1 \\
    Logo: 5+5+1+2+1 = 14
  \end{align*}

  Desta forma, nosso algoritmo considera sendo melhores as soluções com valores
  de aptidão menores.

  \subsection{Crossover e mutação}
  Crossover e mutação são dois operadores genéticos usados nos Algoritmos
  Genéticos. O primeiro consiste em misturar partes de dois cromossomos para
  gerar o novo indivíduo. O segundo consiste em alterar alguns genes do nosso
  cormossomo para gerar um novo infivíduo.

  No presente trabalho, o crossover e a mutação estão intrinsecamente ligados,
  pois não temos genes ativado ou não dependendo dos bits presentes em uma
  posição do cormossomo, mas temos n opções diferentes para cada gene e estes
  não podem se repetir em um cromossomo.

  




\end{document}
